\documentclass{article}

\usepackage[a4paper, total={6in, 8in}]{geometry}
\usepackage{amsmath}
\usepackage{enumerate}
\usepackage{mathrsfs}
\usepackage{amssymb}

\setlength\parindent{0pt}

\newcommand{\workitem}[4]{\item \textsc{#1} (#4) \vskip -3pt #2\vskip -3pt #3}
\newcommand{\exercise}[1]{\item[\textbf{#1}]}
\newcommand{\solution}{\vskip 1.5em}
\newcommand{\set}[1]{\left\{\,#1\,\right\}}
\newcommand{\suchthat}{\,\big|\,}
\newcommand{\divopt}{\,\big|\,}
\newcommand{\R}{\mathbb{R}}
\newcommand{\inv}{^{-1}}
\newcommand{\abs}[1]{\left|#1\right|}
\DeclareMathOperator{\incomp}{||}


\pagestyle{empty}

\begin{document}
\begin{center}

\textbf{Probablistic Combinatorial Games}

\begin{itemize}
    \exercise{Overview.} Probablistic combinatorial games (PCGs) are an extension of
    combinatorial games allowing probabilistic moves to occur.  In particular,
    PCGs are two player games where each player takes turn making moves.  Each player has \textit{complete information}, knowing all moves
    available to them and to their opponents.  Games are won using the \textit{normal play} convention,
    where the last player to make a move is the winner of the game.  The key difference in PCGs as
    opposed to Classic combinatorial games (CCGs) is that rather than selecting a from an array of game options,
    players select a distribution of possible game states, after which a state is chosen at random from the distribution.

    \solution
    \exercise{Definition.}
    We define a game $G$ as
    $$
        G = \set{G_L\divopt G_R}\,.
    $$
    where $G_L$ is a typical left option of $G$ and $G_R$ is a typical right option of $G$.
    As stated previously, $G_L$ and $G_R$ are not single game states, but rather a distribution of the form
    $$
        D = [ (D_1, p_1), (D_2, p_2), \ldots, (D_n, p_n) ]\,,\quad\text{where $\sum_i p_i = 1$.}
    $$
    The states $D_i$ are referred to as a \textit{typical option} of $D$ and $p_i$ is a \textit{typical probability} of $D$.

    A game is played by either the \textit{\textbf{L}eft} player or the \textit{\textbf{R}ight} player selecting
    one of their options distributions, and the game state is then chosen at random from the distribution.
    A player loses when they cannot make a move.

    Note that this definition encompasses the theory of CCGs, since we can merely let $p_1 = 1$.  This means
    that all the games we are used to appear, including the zero and all of the dyadic rationals.

    For notational convenience we will treat a game $G$ as a distribution $[ (G, 1) ]$ if it is
    used where a distribution is required, and simiarly a distribution $D$ of the form
    $D = [(G, 1)]$ will be treated as a game if it is used where a game is required.


    \solution
    \exercise{Algebra.}
    The sum of two games $G$ and $H$ is defined as
    $$
        G + H = \set{ G_L + H, G + H_L \divopt G_R + H, G + H_R }\,.
    $$

    The sum of two distributions $D = [(D_i, p_i)]$ and $F = [(F_j, q_j)]$ is defined as
    $$
        D + F = [(D_i + F_j, \sum_{\substack{k,l \\ D_i + P_j = D_k + F_l}} p_k q_l)]\,.
    $$

    The sum of a distribution $D$ and a game $G$ is defined as
    $$
        D + G = [(D_i + G, p_i)] \quad \text{and} \quad G + D = [(G + D_i, p_i)]
    $$
    $$
        -G = \set{ -G_R \divopt -G_L }
    $$
    $$
        -D = [(-D_i, p_i)]
    $$
    $$
    G - H = G + (-H) = \set{ G_L - H, G - H_R \divopt G_R - H, G - H_L }
    $$

    \solution
    \exercise{$P_L$ and $P_R$.}
    The $P_L$ and $P_R$ functions are defined as follows
    $$
    \begin{aligned}
        P_L(G) &= \max\set{0, P_R(G_L)} &\quad P_R(G) &= \min\set{1, P_L(G_R) }  \\
        P_L(D) &= \sum_{i} p_i P_L(G_i) &\quad P_R(D) &= \sum_{i} p_i P_R(G_i) \\
    \end{aligned}
    $$

    \solution
    \exercise{Ordering.}
    We can use $P_L$ and $P_R$ as indicator functions to define $G > 0$, $G < 0$, $G = 0$, and $G \incomp 0$.

    $$
    \begin{tabular}{c c c}
        \hline
        $P_L(G)$ & $P_R(G)$ & \\
        \hline
        1 & 1 & $G > 0$ \\
        0 & 1 & $G = 0$ \\
        1 & 0 & $G \incomp 0$ \\
        0 & 0 & $G < 0$ \\
        \hline
    \end{tabular}
    $$

    Note that $G \ge 0$ iff $P_R(G) = 1$ and $G \le 0$ iff $P_L(G) = 0$.  It follows that $G \le H$ iff
    $P_L(G - H) = 0$.  Using the definitions of $P_L$ and $P_R$, we can derive the usual definition of $G \le H$
    for combinatorial games.
    $$
        G \le H \iff P_L(G - H) = 0 \iff \max\set{0, P_R(G_L - H), P_R(G - H_R)} = 0\,.
    $$
    That is no $P_R(G_L - H)$ nor $P_R(G - H_R)$ is greater than 0, so no $G_L \ge H$ and no $G \ge H_R$
\end{itemize}

\end{center}
\end{document}
